%cinza pro baixa importancia
\definecolor{gray}{gray}{0.75}

\newenvironment{funcionalidade}[1]
%ANTES
{
	% ---
	% DESCRIÇÃO
	% ---
	\newcommand{\descricao}[1]{
		\index{Descrição}
		\section[Descrição]{\protect\hyperlink{tocsection.\thesection}{Descrição}}

		##1
	}

	% ---
	% IMPORTÂNCIA
	% ---
	%alta
	%media
	%baixa
	\newcommand{\importancia}[1]{
		\textbf{Importância:} 
		\ifstrequal{##1}{alta}{\index{Importância!Alta}\colorbox{red}{\color{white}\textbf{Alta}}}{}
		
		\ifstrequal{##1}{media}{\index{Importância!Média}\colorbox{yellow}{\textbf{Média}}}{}
		\ifstrequal{##1}{baixa}{\index{Importância!Baixa}\colorbox{gray}{Baixa}}{}
	}

	% ---
	% Condições
	% ---

	% \condicao{pre}{pos}

	\newcommand{\condicao}[2]{
		\index{Condições}
		\section[Condições]{\protect\hyperlink{tocsection.\thesection}{Condições}}
		
		\begin{table}[htb]
			\captionsetup{font=scriptsize}
			\begin{tabular}{ p{3cm}  p{12cm} }
				\hline \\
				\textbf{Pré-Condições} & ##1 \\ \\
				\textbf{Pós-Condições} & ##2 \\
				\\ \hline
			\end{tabular}
			\caption[Condições da Funcionalidade #1]{Condições da Funcionalidade #1}
		\end{table}
	}

	% ---
	% FLUXO
	% ---
	\newenvironment{fluxo}[0]
	%ANTES
	{

		% --
		% esses dois renewcommand são para enumerar 	1.
							%		1.1
							%		1.2
							%	2.

		\renewcommand{\labelenumii}{\theenumii}
		\renewcommand{\theenumii}{\theenumi.\arabic{enumii}.}
		% ---------------------------------------------------

		\newcommand{\f}[0]{
			\item 
		}

		\newcommand{\subf}[0]{
			\begin{enumerate}
		}

		\newcommand{\voltaf}[0]{
			\end{enumerate}
		}

		\section[Fluxo de Execução]{\protect\hyperlink{tocsection.\thesection}{Fluxo de Execução}}
		\index{Fluxo de Execução}
		\begin{enumerate}
	}
	%DEPOIS
	{
		\end{enumerate}
	}

	\newcommand{\atores}[1]{
		\index{Atores}
		\section[Atores]{\protect\hyperlink{tocsection.\thesection}{Atores}}
		##1
	}

	\chapter[#1]{\protect\hyperlink{tocchapter.\thechapter}{#1}}
}
%DEPOIS
{

}
