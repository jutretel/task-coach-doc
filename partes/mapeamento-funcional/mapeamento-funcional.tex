\part{Mapeamento Funcional}

% ---
% CRIAR NOVA TAREFA
% ---
\begin{funcionalidade}{Criar nova tarefa}

	%--------------------------------------------------
	\descricao{O usuário pode criar uma nova tarefa e atribuir características a ela: nome, prioridade, datas (início, prazo, datas reais de início e conclusão, lembretes). Pode adicionar também outras tarefas como pré-requisitos, adicionar uma porcentagem de andamento, adicionar a tarefa a uma categoria, adicionar um orçamento, notas, anexos e alterar a aparência da tarefa.}

	%--------------------------------------------------
	\importancia{alta}

	%--------------------------------------------------
	\condicao
	{Não tem pré-condições.} %pré-condição
	{Uma nova tarefa é adicionada ao projeto.} %pós-condição

	%--------------------------------------------------
	\begin{fluxo}
		\f Na tela principal do programa, clicar no ícone de nova tarefa;
			\subf
			\f Uma nova tarefa também poderá ser adicionada clicando no botão INSERT do teclado;
			\voltaf
		\f Uma nova janela irá abrir, com diversas abas para edição da nova tarea;
		\f Após adicionar todas as informações, clicar em Fechar.
	\end{fluxo}

	%--------------------------------------------------
	\atores{Apenas o usuário.}

\end{funcionalidade}

% ---
% EDITAR TAREFA
% ---
\begin{funcionalidade}{Editar Tarefa}

	%--------------------------------------------------
	\descricao{O usuário pode selecionar uma tarefa existente e editar as características dela.}

	%--------------------------------------------------
	\importancia{media}

	%--------------------------------------------------
	\condicao
	{Devem existir tarefas posteriormente criadas.} %pré-condição
	{A tarefa selecionada agora terá as novas características.} %pós-condição

	%--------------------------------------------------
	\begin{fluxo}
		\f Na tela principal, onde todas as tarefas estão sendo listadas, selecionar uma das tarefas;
		\f Clicar no botão de edição logo acima da lista de tarefas;
			\subf
			\f A opção de edição poderá ser aberta com um duplo clique na tarefa desejada;
			\voltaf
		\f Uma nova janela será aberta com as opões de edição;
		\f Depois de alterar a tarefa, clicar em Fechar.
	\end{fluxo}

	%--------------------------------------------------
	\atores{Apenas o usuário.}

\end{funcionalidade}

% ---
% Criar novo subitem tarefa
% ---
\begin{funcionalidade}{Criar subtarefas}

	%--------------------------------------------------
	\descricao{O usuário pode criar subtarefas em outras tarefas principais.}

	%--------------------------------------------------
	\importancia{alta}

	%--------------------------------------------------
	\condicao
	{Devem existir tarefas principais para adicinar subitens.} %pré-condição
	{Um novo subitem será criado relacionado a tarefa ou subtarefa principal.} %pós-condição

	%--------------------------------------------------
	\begin{fluxo}
		\f Na tela principal, selecionar a tarefa ou subtarefa que será relacionada ao novo subitem;
		\f Clicar no ícone de novo subitem logo acima da lista de tarefas;
			\subf
			\f Um novo subitem também poderá ser adicionado através do comando CRTL+SHIFT+N;		
			\voltaf
		\f Uma nova janela será aberta com as características do novo subitem;
		\f Após atribuir as características do subitem, clicar em Fechar.
	\end{fluxo}

	%--------------------------------------------------
	\atores{Apenas o usuário.}

\end{funcionalidade}

% ---
% Excluir Tarefa
% ---
\begin{funcionalidade}{Excluir subtarefa}

	%--------------------------------------------------
	\descricao{O usuário poderá excluir tarefas.}

	%--------------------------------------------------
	\importancia{media}

	%--------------------------------------------------
	\condicao
	{Devem existir tarefas para serem excluídas.} %pré-condição
	{A tarefa selecionada deixará de existir no projeto do usuário.} %pós-condição

	%--------------------------------------------------
	\begin{fluxo}
		\f Na tela principal, selecionar a tarefa a ser excluída;
		\f Clicar no ícone de exclusão;
			\subf
			\f A tarefa também poderá ser excluída ao clicar o botão de DELETE no teclado.
			\voltaf
		\f Se a tarefa possuir subitens, todos serão excluídos após a ação.
	\end{fluxo}

	%--------------------------------------------------
	\atores{Apenas o usuário.}

\end{funcionalidade}


% ---
% Salvar projeto
% ---
\begin{funcionalidade}{Salvar Projeto}

	%--------------------------------------------------
	\descricao{O usuário poderá salvar seus projetos contendo uma lista de tarefas.}

	%--------------------------------------------------
	\importancia{alta}

	%--------------------------------------------------
	\condicao
	{O usuário deve ter alterado ou criado um projeto para salvar.} %pré-condição
	{Se o projeto for novo, um novo arquivo será adicionado na pasta selecionada pelo usuário. Se o projeto já existir, as alterações feitas pelo usuário serão salvas no mesmo arquivo (ou num novo arquivo).} %pós-condição

	%--------------------------------------------------
	\begin{fluxo}
		\f Com o projeto aberto, ir no menu Arquivo, no canto superior esquerdo;
		\f Selecionar:
			\subf
			\f Salvar, caso seja um novo projeto ou um projeto que foi alterado;
			\f Salvar como, caso o usuário tenha alterado um projeto e queira criar um novo arquivo e alterar o atual;
			\f Salvar tarefas selecionadas para novo arquivo de tarefas, caso queira criar um novo arquivo contendo algumas tarefas (mais de uma tarefa poderá ser selecionada segurando a tecla CTRL ao clicar);
			\voltaf
		\f Caso o projeto seja novo ou o usuário tenha selecionado Salvar Como ou Salvar Tarefas Selecionadas, uma nova janela abrirá;
		\f Escolher o nome do projeto;
		\f Escolher a pasta onde deseja-se salvar o projeto;
		\f Clicar em Salvar.
	\end{fluxo}

	%--------------------------------------------------
	\atores{Apenas o usuário.}

\end{funcionalidade}


% ---
% Abrir projeto
% ---
\begin{funcionalidade}{Criar subtarefas}

	%--------------------------------------------------
	\descricao{O usuário poderá abrir um projeto existente para alterá-lo ou ver suas caracteristicas.}

	%--------------------------------------------------
	\importancia{alta}

	%--------------------------------------------------
	\condicao
	{É necessário existir um projeto para abri-lo.} %pré-condição
	{O projeto selecionado pelo usuário abrirá na tela principal.} %pós-condição

	%--------------------------------------------------
	\begin{fluxo}
		\f Na tela principal, ir no menu Arquivo e selecionar a opção Abrir;
			\subf
			\f O usuário poderá abrir um projeto clicando no ícone de abrir na tela principal;
			\voltaf
		\f Selecionar a pasta onde se encontra o projeto a ser aberto;
		\f Selecionar o projeto;
		\f Clicar em Abrir.
	\end{fluxo}

	%--------------------------------------------------
	\atores{Apenas o usuário.}

\end{funcionalidade}

% ---
% IMPRIMIR PROJETO
% ---
\begin{funcionalidade}{Imprimir Projeto}

	%--------------------------------------------------
	\descricao{O usuário poderá imprimir uma lista com as tarefas de um projeto.}

	%--------------------------------------------------
	\importancia{baixa}

	%--------------------------------------------------
	\condicao
	{Ter uma impressora conectada ao computador, ter um projeto aberto.} %pré-condição
	{O usuário terá em mãos um documento com a lista de tarefas criadas no projeto, ele também poderá salvar o arquivo que foi impresso como pdf.} %pós-condição

	%--------------------------------------------------
	\begin{fluxo}
		\f Na tela principal, selecionar o ícone de impressão;
			\subf
			\f A impressão será feita de acordo com a visualização atual, podendo ser: uma lista de tarefas, esforço agrupado por mês, colunas visíveis, ordem de classificação, tarefas filtradas, etc;
		\f Uma nova janela será aberta com opções de impressão;
			\subf 
			\f Se o usuário desejar salvar o arquivo impresso, marcar a caixa de Print to File;
			\voltaf 
		\f Depois de configurar a impressão, apertar em Ok.
	\end{fluxo}

	%--------------------------------------------------
	\atores{Apenas o usuário e impressora.}

\end{funcionalidade}


% ---
% Criar nova categoria
% ---
\begin{funcionalidade}{Criar Nova Categoria}

	%--------------------------------------------------
	\descricao{Uma nova categoria passará a existir e as tarefas poderão ser associadas a ela.}

	%--------------------------------------------------
	\importancia{media}

	%--------------------------------------------------
	\condicao
	{Não tem pré-condições.} %pré-condição
	{Uma nova categoria passará a existir e as tarefas poderão ser associadas a ela.} %pós-condição

	%--------------------------------------------------
	\begin{fluxo}
		\f Na janela principal, no lado direito da tela, existe a coluna de categorias;
		\f Selecionar o ícone de nova categoria;
		\f Uma nova janela abrirá com as opções de características da categoria;
		\f Clicar em Fechar quando os detalhes forem preenchidos.
	\end{fluxo}

	%--------------------------------------------------
	\atores{Apenas o usuário.}
\end{funcionalidade}


% ---
% Criar novo subitem de categoria
% ---
\begin{funcionalidade}{Criar Subcategoria}

	%--------------------------------------------------
	\descricao{O usuário poderá subdividir as categorias em subitens.}

	%--------------------------------------------------
	\importancia{media}

	%--------------------------------------------------
	\condicao
	{Ter pelo menos uma categoria já criada.} %pré-condição
	{Uma nova subcategoria será criada e poderá ter tarefas associadas a ela.} %pós-condição

	%--------------------------------------------------
	\begin{fluxo}
		\f Na tela principal, no lado direito, selecionar uma categoria como principal;
		\f Clicar no botão de novo subitem;
		\f Descrever o novo subitem na janela que aparecer;
		\f Clicar em Fechar.
	\end{fluxo}

	%--------------------------------------------------
	\atores{Apenas o usuário.}

\end{funcionalidade}

% ---
% Editar Categoria
% ---
\begin{funcionalidade}{Editar Categoria}

	%--------------------------------------------------
	\descricao{O usuário poderá editar os detalhes de uma categoria.}

	%--------------------------------------------------
	\importancia{media}

	%--------------------------------------------------
	\condicao
	{Ter pelo menos uma categoria já criada.} %pré-condição
	{A categoria selecionada terá suas características alteradas.} %pós-condição

	%--------------------------------------------------
	\begin{fluxo}
		\f Na tela principal, no lado direito, selecionar uma categoria;
		\f Clicar no botão de edição;
			\subf 
			\f A edição poderá ser feita com um duplo clique na categoria desejada;
			\voltaf
		\f Uma nova janela abrirá, com os detalhes da categoria selecionada;
		\f Alterar as características desejadas;
		\f Clicar em Fechar.
	\end{fluxo}

	%--------------------------------------------------
	\atores{Apenas o usuário.}

\end{funcionalidade}

% ---
% Excluir categoria
% ---
\begin{funcionalidade}{Excluir Categoria}

	%--------------------------------------------------
	\descricao{O usuário poderá excluir uma categoria.}

	%--------------------------------------------------
	\importancia{media}

	%--------------------------------------------------
	\condicao
	{Ter pelo menos uma categoria já criada.} %pré-condição
	{A categoria selecionada e suas subcategorias deixarão de existir e todas as tarefas associadas à categoria ficarão sem categoria.} %pós-condição

	%--------------------------------------------------
	\begin{fluxo}
		\f Na tela principal, no lado direito, selecionar uma categoria;
		\f Clicar no botão de excluir;
			\subf 
			\f A exclusão poderá ser feita com o botão DELETE;
			\voltaf
		\f A categoria e todas suas subcategorias deixarão de existir;

	\end{fluxo}

	%--------------------------------------------------
	\atores{Apenas o usuário.}

\end{funcionalidade}

\begin{funcionalidade}{Pesquisar Categoria}

	%--------------------------------------------------
	\descricao{O usuário poderá pesquisar por uma categoria que deseja encontrar mais fácil.}

	%--------------------------------------------------
	\importancia{baixa}

	%--------------------------------------------------
	\condicao
	{Ter categorias criadas.} %pré-condição
	{Se a pesquisa corresponder a uma categoria existente, ela aparecerá, na lista, em primeiro.} %pós-condição

	%--------------------------------------------------
	\begin{fluxo}
		\f Na tela principal, no lado direito, clicar no campo de texto com a lupa à esquerda;
		\f Cliclando na lupa, outras opções de pesquisa aparecerão;
			\subf 
			\f Procurar apenas por categorias;
			\f Procurar diferenciando maiúsculas de minúsculas ;
			\f Incluir subitens na pesquisa;
			\f Procurar por descrições;
			\f Procurar por expressão regular;
			\voltaf
		\f As categorias encontradas aparecerão na tela;

	\end{fluxo}

	%--------------------------------------------------
	\atores{Apenas o usuário.}

\end{funcionalidade}

\begin{funcionalidade}{Mesclar Projetos}

	%--------------------------------------------------
	\descricao{O usuário poderá juntar dois projetos.}

	%--------------------------------------------------
	\importancia{baixa}

	%--------------------------------------------------
	\condicao
	{Ter dois projetos salvos.} %pré-condição
	{O projeto aberto receberá as informações do projeto escolhido para mesclar.} %pós-condição

	%--------------------------------------------------
	\begin{fluxo}
		\f Abrir um projeto ou ter feito alterações num novo projeto;
		\f Na tela principal, clicar no menu Arquivo;
		\f Escolher a opção Mesclar;
		\f Uma nova janela abrirá com as pastas do computador, para selecionar o segundo projeto (que se deseja mesclar com o atual);
		\f Clicar em OK.

	\end{fluxo}

	%--------------------------------------------------
	\atores{Apenas o usuário.}

\end{funcionalidade}

\begin{funcionalidade}{Importar Projetos}

	%--------------------------------------------------
	\descricao{O usuário poderá importar tarefas de outros tipos de arquivos (como txt e CSV).}

	%--------------------------------------------------
	\importancia{baixa}

	%--------------------------------------------------
	\condicao
	{Ter arquivo no formato válido para importação.} %pré-condição
	{O projeto aberto receberá as informações do arquivo escolhido para importar tarefas.} %pós-condição

	%--------------------------------------------------
	\begin{fluxo}
		\f Na tela principal, clicar no menu Arquivo;
		\f Escolher a opção Importar e a extensão do arquivo que será importado;
		\f Uma nova janela abrirá com as pastas do computador, para selecionar o arquivo a ser importado;
			\subf
			\f No caso de txt, cada linha do arquivo será convertido em uma tarefa;
			\voltaf
		\f Clicar em OK.

	\end{fluxo}

	%--------------------------------------------------
	\atores{Apenas o usuário.}

\end{funcionalidade}


\begin{funcionalidade}{Exportar Projetos}

	%--------------------------------------------------
	\descricao{O usuário poderá exportar seus projetos para outro formato de arquivo.}

	%--------------------------------------------------
	\importancia{baixa}

	%--------------------------------------------------
	\condicao
	{Ter um projeto.} %pré-condição
	{Será criado um novo arquivo com outra extensão contendo as informações do projeto.} %pós-condição

	%--------------------------------------------------
	\begin{fluxo}
		\f Ter um projeto aberto;
		\f Na tela principal, clicar no menu Arquivo;
		\f Escolher a opção Exportar e a extensão para qual se deseja exportar;
		\f Uma nova janela abrirá com as pastas do computador, para selecionar o local onde o novo arquivo será salvo;
		\f Clicar em OK.

	\end{fluxo}

	%--------------------------------------------------
	\atores{Apenas o usuário.}

\end{funcionalidade}


\begin{funcionalidade}{Desfazer/Refazer Ações}

	%--------------------------------------------------
	\descricao{O usuário poderá desfazer ou refazer ações.}

	%--------------------------------------------------
	\importancia{baixa}

	%--------------------------------------------------
	\condicao
	{Ter realizado ações.} %pré-condição
	{No caso de desfazer: a última ação realizada será desfeita. No caso de refazer: a ultima ação desfeita será refeita.} %pós-condição

	%--------------------------------------------------
	\begin{fluxo}
		\f No alto da tela principal, clicar no ícone de desfazer;
			\subf
			\f Caso a ação desfeita queira ser refeita, clicar no ícone de refazer.
			\voltaf

	\end{fluxo}

	%--------------------------------------------------
	\atores{Apenas o usuário.}

\end{funcionalidade}