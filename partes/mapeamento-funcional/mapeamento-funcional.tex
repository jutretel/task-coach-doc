\part{Mapeamento Funcional}

% ---
% CRIAR NOVA TAREFA
% ---
\begin{funcionalidade}{Criar nova tarefa}

	%--------------------------------------------------
	\descricao{O usuário pode criar uma nova tarefa e atribuir características a ela: nome, prioridade, datas (início, prazo, datas reais de início e conclusão, lembretes). Pode adicionar também outras tarefas como pré-requisitos, adicionar uma porcentagem de andamento, adicionar a tarefa a uma categoria, adicionar um orçamento, notas, anexos e alterar a aparência da tarefa.}

	%--------------------------------------------------
	\importancia{alta}

	%--------------------------------------------------
	\condicao
	{Não tem pré-condições.} %pré-condição
	{Uma nova tarefa é adicionada ao projeto.} %pós-condição

	%--------------------------------------------------
	\begin{fluxo}
		\f Na tela principal do programa, clicar no ícone de nova tarefa;
			\subf
			\f Uma nova tarefa também poderá ser adicionada clicando no botão INSERT do teclado;
			\voltaf
		\f Uma nova janela irá abrir, com diversas abas para edição da nova tarea;
		\f Após adicionar todas as informações, clicar em Fechar.
	\end{fluxo}

	%--------------------------------------------------
	\atores{Apenas o usuário.}

\end{funcionalidade}

% ---
% EDITAR TAREFA
% ---
\begin{funcionalidade}{Editar Tarefa}

	%--------------------------------------------------
	\descricao{O usuário pode selecionar uma tarefa existente e editar as características dela.}

	%--------------------------------------------------
	\importancia{media}

	%--------------------------------------------------
	\condicao
	{Devem existir tarefas posteriormente criadas.} %pré-condição
	{A tarefa selecionada agora terá as novas características.} %pós-condição

	%--------------------------------------------------
	\begin{fluxo}
		\f Na tela principal, onde todas as tarefas estão sendo listadas, selecionar uma das tarefas;
		\f Clicar no botão de edição logo acima da lista de tarefas;
			\subf
			\f A opção de edição poderá ser aberta com um duplo clique na tarefa desejada;
			\voltaf
		\f Uma nova janela será aberta com as opões de edição;
		\f Depois de alterar a tarefa, clicar em Fechar.
	\end{fluxo}

	%--------------------------------------------------
	\atores{Apenas o usuário.}

\end{funcionalidade}

% ---
% Criar novo subitem tarefa
% ---

\begin{funcionalidade}{Criar subtarefas}

	%--------------------------------------------------
	\descricao{O usuário pode criar subtarefas em outras tarefas principais.}

	%--------------------------------------------------
	\importancia{alta}

	%--------------------------------------------------
	\condicao
	{Devem existir tarefas principais para adicinar subitens.} %pré-condição
	{Um novo subitem será criado relacionado a tarefa ou subtarefa principal.} %pós-condição

	%--------------------------------------------------
	\begin{fluxo}
		\f Na tela principal, selecionar a tarefa ou subtarefa que será relacionada ao novo subitem;
		\f Clicar no ícone de novo subitem logo acima da lista de tarefas;
			\subf
			\f Um novo subitem também poderá ser adicionado através do comando CRTL+SHIFT+N;		
			\voltaf
		\f Uma nova janela será aberta com as características do novo subitem;
		\f Após atribuir as características do subitem, clicar em Fechar.
	\end{fluxo}

	%--------------------------------------------------
	\atores{Apenas o usuário.}

\end{funcionalidade}
