%% abtex2-modelo-relatorio-tecnico.tex, v-1.9.5 laurocesar
%% Copyright 2012-2015 by abnTeX2 group at http://www.abntex.net.br/ 
%%
%% This work may be distributed and/or modified under the
%% conditions of the LaTeX Project Public License, either version 1.3
%% of this license or (at your option) any later version.
%% The latest version of this license is in
%%   http://www.latex-project.org/lppl.txt
%% and version 1.3 or later is part of all distributions of LaTeX
%% version 2005/12/01 or later.
%%
%% This work has the LPPL maintenance status `maintained'.
%% 
%% The Current Maintainer of this work is the abnTeX2 team, led
%% by Lauro César Araujo. Further information are available on 
%% http://www.abntex.net.br/
%%
%% This work consists of the files abntex2-modelo-relatorio-tecnico.tex,
%% abntex2-modelo-include-comandos and abntex2-modelo-references.bib
%%

% ------------------------------------------------------------------------
% ------------------------------------------------------------------------
% abnTeX2: Modelo de Relatório Técnico/Acadêmico em conformidade com 
% ABNT NBR 10719:2011 Informação e documentação - Relatório técnico e/ou
% científico - Apresentação
% ------------------------------------------------------------------------ 
% ------------------------------------------------------------------------

\documentclass[
	% -- opções da classe memoir --
	12pt,				% tamanho da fonte
	openright,			% capítulos começam em pág ímpar (insere página vazia caso preciso)
	oneside,			% não colocar folhas em branco
	a4paper,			% tamanho do papel. 
	% -- opções da classe abntex2 --
	%chapter=TITLE,		% títulos de capítulos convertidos em letras maiúsculas
	%section=TITLE,		% títulos de seções convertidos em letras maiúsculas
	%subsection=TITLE,	% títulos de subseções convertidos em letras maiúsculas
	%subsubsection=TITLE,% títulos de subsubseções convertidos em letras maiúsculas
	% -- opções do pacote babel --
	english,			% idioma adicional para hifenização
	french,				% idioma adicional para hifenização
	spanish,			% idioma adicional para hifenização
	brazil,				% o último idioma é o principal do documento
	]{abntex2}


% ---
% PACOTES
% ---

% ---
% Pacotes fundamentais 
% ---
\usepackage{lmodern}			% Usa a fonte Latin Modern
\usepackage[T1]{fontenc}		% Selecao de codigos de fonte.
\usepackage[utf8]{inputenc}		% Codificacao do documento (conversão automática dos acentos)
\usepackage{indentfirst}		% Indenta o primeiro parágrafo de cada seção.
\usepackage{color}				% Controle das cores
\usepackage{graphicx}			% Inclusão de gráficos
\usepackage{microtype} 			% para melhorias de justificação
\usepackage{caption}			% usado para alterar o tamanho da legenda
% ---
% ---
% Pacotes adicionais, usados no anexo do modelo de folha de identificação
% ---
\usepackage{multicol}
\usepackage{multirow}
% ---
	
% ---
% Pacotes adicionais, usados apenas no âmbito do Modelo Canônico do abnteX2
% ---
\usepackage{lipsum}				% para geração de dummy text
% ---

% ---
% Pacotes de citações
% ---
\usepackage[brazilian,hyperpageref]{backref}	 % Paginas com as citações na bibl
\usepackage[alf]{abntex2cite}	% Citações padrão ABNT

% ---
% Pacote para condicionais
% ---
\usepackage{etoolbox}

% --- 
% CONFIGURAÇÕES DE PACOTES
% --- 

% ---
% Configurações do pacote backref
% Usado sem a opção hyperpageref de backref
\renewcommand{\backrefpagesname}{Citado na(s) página(s):~}
% Texto padrão antes do número das páginas
\renewcommand{\backref}{}
% Define os textos da citação
\renewcommand*{\backrefalt}[4]{
	\ifcase #1 %
		Nenhuma citação no texto.%
	\or
		Citado na página #2.%
	\else
		Citado #1 vezes nas páginas #2.%
	\fi}%
% ---

% ---
% Informações de dados para CAPA e FOLHA DE ROSTO
% ---
\titulo{Documentação do Software Task Coach}

%---
\autor	{%
	Gustavo Yudi Bientinezi Matsuzake
	\and
	Julia Ulson Tretel
	}

\local{Brasil}
\data{Curitiba, 2015}
\instituicao{%
	Universidade Tecnológica Federal do Paraná
	\par
	Departamento Acadêmico de Informática - DAINF
	\par
	Sistemas Legados}
\tipotrabalho{Trabalho Prático}
% O preambulo deve conter o tipo do trabalho, o objetivo, 
% o nome da instituição e a área de concentração 
\preambulo{Trabalho Prático apresentado à disciplina Sistemas Legados
como requisito parcial para aprovação.}
% ---

% ---
% Configurações de aparência do PDF final

% alterando o aspecto da cor azul
\definecolor{blue}{RGB}{41,5,195}

% informações do PDF
\makeatletter
\hypersetup{
     	%pagebackref=true,
		pdftitle={\@title}, 
		pdfauthor={\@author},
    	pdfsubject={\imprimirpreambulo},
	    pdfcreator={LaTeX with abnTeX2},
		pdfkeywords={abnt}{latex}{abntex}{abntex2}{relatório técnico}, 
		colorlinks=true,       		% false: boxed links; true: colored links
    	linkcolor=blue,          	% color of internal links
    	citecolor=blue,        		% color of links to bibliography
    	filecolor=magenta,      		% color of file links
		urlcolor=blue,
		bookmarksdepth=4
}
\makeatother
% --- 

% --- 
% Espaçamentos entre linhas e parágrafos 
% --- 

% O tamanho do parágrafo é dado por:
\setlength{\parindent}{1.3cm}

% Controle do espaçamento entre um parágrafo e outro:
\setlength{\parskip}{0.2cm}  % tente também \onelineskip

% ---
% compila o indice
% ---
\makeindex
% ---

% ----
% Início do documento
% ----
\begin{document}

% Seleciona o idioma do documento (conforme pacotes do babel)
%\selectlanguage{english}
\selectlanguage{brazil}

% Retira espaço extra obsoleto entre as frases.
\frenchspacing 

% ----------------------------------------------------------
% ELEMENTOS PRÉ-TEXTUAIS
% ----------------------------------------------------------
% \pretextual

% ---
% Capa
% ---
\imprimircapa
% ---

% ---
% Folha de rosto
% (o * indica que haverá a ficha bibliográfica)
% ---
\imprimirfolhaderosto*
% ---

% ---
% RESUMO
% ---

% resumo na língua vernácula (obrigatório)
%\setlength{\absparsep}{18pt} % ajusta o espaçamento dos parágrafos do resumo
%\begin{resumo}
% Segundo a \citeonline[3.1-3.2]{NBR6028:2003}, o resumo deve ressaltar o
% objetivo, o método, os resultados e as conclusões do documento. A ordem e a extensão
% destes itens dependem do tipo de resumo (informativo ou indicativo) e do
% tratamento que cada item recebe no documento original. O resumo deve ser
% precedido da referência do documento, com exceção do resumo inserido no
% próprio documento. (\ldots) As palavras-chave devem figurar logo abaixo do
% resumo, antecedidas da expressão Palavras-chave:, separadas entre si por
% ponto e finalizadas também por ponto.
%
% \noindent
% \textbf{Palavras-chaves}: latex. abntex. editoração de texto.
%\end{resumo}
% ---

% ---
% inserir lista de ilustrações
% ---
%\pdfbookmark[0]{\listfigurename}{lof}
%\listoffigures*
%\cleardoublepage
% ---

% ---
% inserir lista de tabelas
% ---
\pdfbookmark[0]{\listtablename}{lot}
\listoftables*
\cleardoublepage
%% ---

% ---
% inserir lista de abreviaturas e siglas
% ---
%\begin{siglas}
%  \item[ABNT] Associação Brasileira de Normas Técnicas
%  \item[abnTeX] ABsurdas Normas para TeX
%\end{siglas}
% ---

% ---
% inserir lista de símbolos
% ---
%\begin{simbolos}
%  \item[$ \Gamma $] Letra grega Gama
%  \item[$ \Lambda $] Lambda
%  \item[$ \zeta $] Letra grega minúscula zeta
%  \item[$ \in $] Pertence
%\end{simbolos}
% ---

% ---
% inserir o sumario
% ---
\pdfbookmark[0]{\contentsname}{toc}
\tableofcontents*
\cleardoublepage
% ---


% ----------------------------------------------------------
% ELEMENTOS TEXTUAIS
% ----------------------------------------------------------
\textual

% ----------------------------------------------------------
% Introdução (exemplo de capítulo sem numeração, mas presente no Sumário)
% ----------------------------------------------------------
%\chapter*[Introdução]{Introdução}
%\addcontentsline{toc}{chapter}{Introdução}

% ----------------------------------------------------------
% PARTES DA DOCUMENTAÇÃO
% ----------------------------------------------------------
%cinza pro baixa importancia
\definecolor{gray}{gray}{0.75}

\newenvironment{funcionalidade}[1]
%ANTES
{
	% ---
	% DESCRIÇÃO
	% ---
	\newcommand{\descricao}[1]{
		\index{Descrição}
		\section{Descrição}

		##1
	}

	% ---
	% IMPORTÂNCIA
	% ---
	%alta
	%media
	%baixa
	\newcommand{\importancia}[1]{
		\index{Importância}
		\textbf{Importância:} 
		\ifstrequal{##1}{alta}{\colorbox{red}{\color{white}\textbf{Alta}}}{}
		\ifstrequal{##1}{media}{\colorbox{yellow}{\textbf{Média}}}{}
		\ifstrequal{##1}{baixa}{\colorbox{gray}{Baixa}}{}
	}

	% ---
	% Condições
	% ---

	% \condicao{pre}{pos}

	\newcommand{\condicao}[2]{
		\index{Condições}
		\section{Condições}
		
		\begin{table}[htb]
			\captionsetup{font=scriptsize}
			\begin{tabular}{ p{3cm}  p{12cm} }
				\hline \\
				\textbf{Pré-Condições} & ##1 \\ \\
				\textbf{Pós-Condições} & ##2 \\
				\\ \hline
			\end{tabular}
			\caption[Condições da Funcionalidade #1]{Condições da Funcionalidade #1}
		\end{table}
	}

	% ---
	% FLUXO
	% ---
	\newenvironment{fluxo}[0]
	%ANTES
	{

		% --
		% esses dois renewcommand são para enumerar 	1.
							%		1.1
							%		1.2
							%	2.

		\renewcommand{\labelenumii}{\theenumii}
		\renewcommand{\theenumii}{\theenumi.\arabic{enumii}.}
		% ---------------------------------------------------

		\newcommand{\f}[0]{
			\item 
		}

		\newcommand{\subf}[0]{
			\begin{enumerate}
		}

		\newcommand{\voltaf}[0]{
			\end{enumerate}
		}

		\section{Fluxo de Execução}
		\index{Fluxo de Execução}
		\begin{enumerate}
	}
	%DEPOIS
	{
		\end{enumerate}
	}

	\newcommand{\atores}[1]{
		\index{Atores}
		\section{Atores}
		##1
	}

	\chapter{#1}
}
%DEPOIS
{

}

\part{Mapeamento Funcional}

% ---
% CRIAR NOVA TAREFA
% ---
\begin{funcionalidade}{Criar nova tarefa}

	%--------------------------------------------------
	\descricao{O usuário pode criar uma nova tarefa e atribuir características a ela: nome, prioridade, datas (início, prazo, datas reais de início e conclusão, lembretes). Pode adicionar também outras tarefas como pré-requisitos, adicionar uma porcentagem de andamento, adicionar a tarefa a uma categoria, adicionar um orçamento, notas, anexos e alterar a aparência da tarefa.}

	%--------------------------------------------------
	\importancia{alta}

	%--------------------------------------------------
	\condicao
	{Não tem pré-condições.} %pré-condição
	{Uma nova tarefa é adicionada ao projeto.} %pós-condição

	%--------------------------------------------------
	\begin{fluxo}
		\f Na tela principal do programa, clicar no ícone de nova tarefa;
			\subf
			\f Uma nova tarefa também poderá ser adicionada clicando no botão INSERT do teclado;
			\voltaf
		\f Uma nova janela irá abrir, com diversas abas para edição da nova tarea;
		\f Após adicionar todas as informações, clicar em Fechar.
	\end{fluxo}

	%--------------------------------------------------
	\atores{Apenas o usuário.}

\end{funcionalidade}

% ---
% EDITAR TAREFA
% ---
\begin{funcionalidade}{Editar Tarefa}

	%--------------------------------------------------
	\descricao{O usuário pode selecionar uma tarefa existente e editar as características dela.}

	%--------------------------------------------------
	\importancia{media}

	%--------------------------------------------------
	\condicao
	{Devem existir tarefas posteriormente criadas.} %pré-condição
	{A tarefa selecionada agora terá as novas características.} %pós-condição

	%--------------------------------------------------
	\begin{fluxo}
		\f Na tela principal, onde todas as tarefas estão sendo listadas, selecionar uma das tarefas;
		\f Clicar no botão de edição logo acima da lista de tarefas;
			\subf
			\f A opção de edição poderá ser aberta com um duplo clique na tarefa desejada;
			\voltaf
		\f Uma nova janela será aberta com as opões de edição;
		\f Depois de alterar a tarefa, clicar em Fechar.
	\end{fluxo}

	%--------------------------------------------------
	\atores{Apenas o usuário.}

\end{funcionalidade}

% ---
% Criar novo subitem tarefa
% ---

\begin{funcionalidade}{Criar subtarefas}

	%--------------------------------------------------
	\descricao{O usuário pode criar subtarefas em outras tarefas principais.}

	%--------------------------------------------------
	\importancia{alta}

	%--------------------------------------------------
	\condicao
	{Devem existir tarefas principais para adicinar subitens.} %pré-condição
	{Um novo subitem será criado relacionado a tarefa ou subtarefa principal.} %pós-condição

	%--------------------------------------------------
	\begin{fluxo}
		\f Na tela principal, selecionar a tarefa ou subtarefa que será relacionada ao novo subitem;
		\f Clicar no ícone de novo subitem logo acima da lista de tarefas;
			\subf
			\f Um novo subitem também poderá ser adicionado através do comando CRTL+SHIFT+N;		
			\voltaf
		\f Uma nova janela será aberta com as características do novo subitem;
		\f Após atribuir as características do subitem, clicar em Fechar.
	\end{fluxo}

	%--------------------------------------------------
	\atores{Apenas o usuário.}

\end{funcionalidade}

\part{Diagrama De Classes}

\diagramadeclasse{Base (1/2)}{base_1.pdf}{width=.95\paperwidth}
\section{Descrição}
	Esse diagrama contém classes genéricas de atributos e a classe Filter que será usada para filtrar categorias, tarefas e pesquisas realizadas.

	Estão relacionadas com esse diagrama as funcionalidades 12 (Pesquisar Categoria) e 22 (Inserir Tarefa em uma ou mais Categorias).

\diagramadeclasse{Base (2/2)}{base_2.pdf}{width=.95\paperwidth}
\section{Descrição}
	Esse diagrama contém as classes relacionadas à objetos genéricos e a classe Sorter, que tem função de decorar listas e ordenar seus componentes. 

\diagramadeclasse{Tarefa}{task_cortado.pdf}{width=.95\paperwidth}
\section{Descrição}
	Esse diagrama possuí as classes básicas relacionadas à tarefas, além de classes relacionadas a lista de tarefas que serão criadas durante a execução do programa. 
	
	Estão relacionadas com esse diagrama as funcionalidades 1 (Criar Nova Tarefa), 2 (Editar Tarefa), 3 (Criar Subtarefas), 4 (Excluir Subtarefas). 

\diagramadeclasse{Nova tarefa}{new_task_cortado.pdf}{width=.95\paperwidth}
\section{Descrição}
	Esse diagrama contém classes utilizadas no momento de criação de novas tarefas, além de classes que definem tarefas como parte de uma categoria e como dependência ou pre-requisito de outras tarefas. 
	
	Estão relacionadas com esse diagrama as funcionalidades 1 (Criar Nova Tarefa), 3 (Criar Subtarefa), 20 (Selecionar Pré-Requisitos) e 22 (Inserir Tarefa e uma ou mais Categorias).

\diagramadeclasse{Categoria (1/2)}{category_cortado.pdf}{width=.95\paperwidth}
\section{Descrição}
	Nesse diagrama estão as classes relacionadas às categorias, além das classes que representam as categorias como listas, o filtro e o ordenador de categorias.
	
	Estão relacionadas com esse diagrama as funcionalidades 8 (Criar Nova Categoria), 9 (Criar Subcategoria), 10 (Editar Categoria), 11 (Excluir Categoria), 12 (Pesquisar Categoria) e 22 (Inserir Tarefa em uma ou mais Categorias).

\diagramadeclasse{Categoria (2/2)}{catogory_2_cortado.pdf}{width=.95\paperwidth}
\section{Descrição}
	Nesse diagrama estão algumas classes básicas que chamam os comandos relacionados a categorias, como resetar o filtro e adicionar nova categoria.
	
	Estão relacionadas com esse diagrama as funcionalidades 8 (Criar Nova Categoria) e 12 (Pesquisar Categoria).

\diagramadeclasse{Configurações}{config_cortado.pdf}{width=.95\paperwidth}
\section{Descrição}
	Nesse diagrama estão representadas as classes de configurações genéricas, além de classes auxiliares (que analisam as configurações e as opções de aplicação).

\diagramadeclasse{Data}{date_cortado.pdf}{width=.95\paperwidth}
\section{Descrição}
	Nesse diagrama estão as classes que contém todo tipo de informação relacionada a datas e períodos.
	
	Estão relacionadas com esse diagrama as funcionalidades 1 (Criar Nova Tarefa), 2 (Editar Tarefa), 3 (Criar Subtarefas), 17 (Iniciar Monitoramento de Esforço), 18 (Parar Monitoramento de Esforço), 19 (Alterar Datas da Tarefa) e 21 (Alterar Andamento da Tarefa).

\diagramadeclasse{Esforço (1/2)}{effort_2.pdf}{width=.95\paperwidth}
\section{Descrição}
	O diagrama mostra as classes que se relacionam com a opção de esforço de tarefas no programa e a listagem de esforços existentes.
	
	Estão relacionadas com esse diagrama as funcionalidades 17 (Iniciar Monitoramento de Esforço) e 18 (Parar Monitoramento de Esforço).

\diagramadeclasse{Esforço (2/2)}{effort_1.pdf}{width=.95\paperwidth}
\section{Descrição}
	Esse diagrama complementa o anterior, e contém classes mas específicas sobre a opção de esforço do programa.
	
	Estão relacionadas com esse diagrama as funcionalidades 17 (Iniciar Monitoramento de Esforço) e 18 (Parar Monitoramento de Esforço). 

\diagramadeclasse{Comandos de Entrada e Saída}{io_commands_cortado.pdf}{width=.95\paperwidth}
\section{Descrição}
	Nesse diagrama estão representadas as classes relacionadas aos comandos de entrada e saída, ou seja, comandos de exportação para arquivos e importação de arquivos, além de salvar, abrir e mesclar projetos.
	
	Estão relacionadas com esse diagrama as funcionalidades 5 (Salvar Projeto), 6 (Abrir Projeto), 13 (Mesclar Projetos), 14 (Importar Projetos) e 15 (Exportar Projetos).

\diagramadeclasse{Notas}{note_cortado.pdf}{width=.95\paperwidth}
\section{Descrição}
	Nesse diagrama simples estão representadas as classes relacionadas às notas que podem ser adicionadas a tarefas ou categorias.
	
	Está relacionada com esse diagrama a funcionalidade 24 (Inserir Notas em Tarefas e em Categorias).

\diagramadeclasse{Camada de Persistência (1/2)}{persistence_1.pdf}{width=.95\paperwidth}
\section{Descrição}
	Esse diagrama contém algumas classes relacionadas ao arquivo criado como projeto. Funções como salvar automáticamente e fazer backup de um projeto estão presentes nesse diagrama.
	
	Estão relacionadas com esse diagrama as funcionalidades 5 (Salvar Projeto), 6 (Abrir Projeto) e 15 (Exportar Projetos).

\diagramadeclasse{Camada de Persistência (2/2)}{persistence_2.pdf}{width=.95\paperwidth}
\section{Descrição}
	Outras classes relacionadas aos arquivos do programa estão presentes nesse diagrama. Classe de arquivos temporarios e de backup automático estão no diagrama.
	
	Estão relacionadas com esse diagrama as funcionalidades 5 (Salvar Projeto), 6 (Abrir Projeto) e 15 (Exportar Projetos).

\diagramadeclasse{Linha do tempo}{timeline_cortado.pdf}{width=.95\paperwidth}
\section{Descrição}
	Esse diagrama contém as classes relacionadas com a timeline principal do programa, onde as tarefas irão aparecer e serão gerenciadas.
	
	Está relacionada com esse diagrama a funcionalidade 26 (Alterar a Aparência da Tarefa ou Categoria).

\diagramadeclasse{Desfazer e Refazer}{undo_redo_cortado.pdf}{width=.95\paperwidth}
\section{Descrição}
	Essas duas classes implementam as funções de desfazer e refazer ações no programa.
	
	Está relacionada com esse diagrama a funcionalidade 16 (Desfazer/Refazer Ações).

	

\part{Diagrama De Entidade Relacionamento}

\part{Diagrama De Atividade}


% ---
% Finaliza a parte no bookmark do PDF
% para que se inicie o bookmark na raiz
% e adiciona espaço de parte no Sumário
% ---
\phantompart

% ----------------------------------------------------------
% ELEMENTOS PÓS-TEXTUAIS
% ----------------------------------------------------------
\postextual

% ----------------------------------------------------------
% Referências bibliográficas
% ----------------------------------------------------------
%\bibliography{abntex2-modelo-references}

%---------------------------------------------------------------------
% INDICE REMISSIVO
%---------------------------------------------------------------------

\phantompart
\printindex

\end{document}
